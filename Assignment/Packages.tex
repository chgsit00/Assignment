% Anpassung des Seitenlayouts --------------------------------------------------
%   siehe Seitenstil.tex
% ------------------------------------------------------------------------------
	\usepackage[
	    automark, % Kapitelangaben in Kopfzeile automatisch erstellen
	    headsepline, % Trennlinie unter Kopfzeile
	    ilines % Trennlinie linksb�ndig ausrichten
	]{scrpage2}

% Basic Packages
	\usepackage{framed}
	\usepackage[center]{subfigure}
	\usepackage{amsmath}
	\usepackage{tikz}
	\usepackage{pdfpages} 

% Anpassung an Landessprache ---------------------------------------------------
	\usepackage[english]{babel}
	
%F�r Grafiken	
	\usepackage{graphicx}
	\usepackage{caption}

% Umlaute ----------------------------------------------------------------------
%   Umlaute/Sonderzeichen wie ���� direkt im Quelltext verwenden (CodePage).
%   Erlaubt automatische Trennung von Worten mit Umlauten.
% ------------------------------------------------------------------------------
	\usepackage[latin1]{inputenc}
	\usepackage[T1]{fontenc}
	\usepackage{textcomp} % Euro-Zeichen etc.

% Schrift ----------------------------------------------------------------------
	\usepackage{lmodern} % bessere Fonts
	\usepackage{relsize} % Schriftgr��e relativ festlegen

% Grafiken ---------------------------------------------------------------------
% Einbinden von JPG-Grafiken erm�glichen
%\usepackage[pdftex]{graphicx} 
% hier liegen die Bilder des Dokuments
	\graphicspath{{Bilder/}}

% Befehle aus AMSTeX f�r mathematische Symbole z.B. \boldsymbol \mathbb --------
	\usepackage{amsmath,amsfonts}

% f�r Index-Ausgabe mit \printindex --------------------------------------------
	%\usepackage{makeidx}

% Einfache Definition der Zeilenabst�nde und Seitenr�nder etc. -----------------
	\usepackage{setspace}
	\usepackage{geometry}
	\usepackage{color}
% Symbolverzeichnis ------------------------------------------------------------
%   Symbolverzeichnisse bequem erstellen. Beruht auf MakeIndex:
%     makeindex.exe %Name%.nlo -s nomencl.ist -o %Name%.nls
%   erzeugt dann das Verzeichnis. Dieser Befehl kann z.B. im TeXnicCenter
%   als Postprozessor eingetragen werden, damit er nicht st�ndig manuell
%   ausgef�hrt werden muss.
%   Die Definitionen sind ausgegliedert in die Datei "Glossar.tex".
% ------------------------------------------------------------------------------
	\usepackage[intoc]{nomencl}
	%\let\abbrev\nomenclature
	%\renewcommand{\nomname}{Symbolverzeichnis}
	%\setlength{\nomlabelwidth}{.25\hsize}
	%\renewcommand{\nomlabel}[1]{#1 \dotfill}
	%\setlength{\nomitemsep}{-\parsep}

% zum Umflie�en von Bildern ----------------------------------------------------
	\usepackage{floatflt}


% zum Einbinden von Programmcode -----------------------------------------------
	\usepackage{listings}
	\usepackage{xcolor} 
	\definecolor{hellgelb}{rgb}{1,1,0.9}
	\definecolor{colKeys}{rgb}{0,0,1}
	\definecolor{colIdentifier}{rgb}{0,0,0}
	\definecolor{colComments}{rgb}{1,0,0}
	\definecolor{colString}{rgb}{0,0.5,0}
	
%Einstellungen des Codesstyles, Farbe ect.:
	\lstset{
		captionpos=top,
		language=Python,
	    float=hbp,
	    basicstyle=\ttfamily\color{black}\small\smaller,
	    identifierstyle=\color{colIdentifier},
	    keywordstyle=\color{colKeys},
	    stringstyle=\color{colString},
	    commentstyle=\color{colComments},
	    columns=flexible,
	    tabsize=5,
	    frame=single,
	    extendedchars=true,
	    showspaces=false,
	    showstringspaces=false,
	    numbers=left,
	    numberstyle=\tiny,
	    breaklines=true,
	    backgroundcolor=\color{hellgelb},
	    breakautoindent=true
	}

% URL verlinken, lange URLs umbrechen etc. -------------------------------------
	\usepackage{url}

% wichtig f�r korrekte Zitierweise(Habe ich nichtmehr gebraucht) ---------------------------------------------
	%\usepackage[numbers]{natbib} 
	%\usepackage{natbib}
	%\setcitestyle{numbers}
	%\usepackage{cite}

% Literaturverzeichnis Layout definieren
	\usepackage[style=alphabetic, backend=bibtex, block=space, pagetracker=true, backref=true, sorting=none,sortcites =true]{biblatex}
	\setlength{\bibitemsep}{1em}     % Abstand zwischen den Literaturangaben
	\setlength{\bibhang}{2em}        % Einzug nach jeweils erster Zeile



% PDF-Optionen -----------------------------------------------------------------
	\usepackage[
	    bookmarks,
	    bookmarksopen=true,
	    colorlinks=true,
% diese Farbdefinitionen zeichnen Links im PDF farblich aus. Gut f�r PDF!
	 %   linkcolor=red, % einfache interne Verkn�pfungen
	  %  anchorcolor=black,% Ankertext
	  %  citecolor=blue, % Verweise auf Literaturverzeichniseintr�ge im Text
	  %  filecolor=magenta, % Verkn�pfungen, die lokale Dateien �ffnen
	  %  menucolor=red, % Acrobat-Men�punkte
	  %  urlcolor=cyan, 
%Neue Einstellungen der farben!!!Verwendeung zum Drucken! Da schwarz gew�nscht ist!
		linkcolor=black,
	    anchorcolor=black,% Ankertext
	    citecolor=black, % Verweise auf Literaturverzeichniseintr�ge im Text
	    filecolor=black, % Verkn�pfungen, die lokale Dateien �ffnen
	    menucolor=black, % Acrobat-Men�punkte
	    urlcolor=black,
	    frenchlinks=false,
	    breaklinks = false,      
	  %   backref,
	    plainpages=false, % zur korrekten Erstellung der Bookmarks
	    pdfpagelabels, % zur korrekten Erstellung der Bookmarks
	   % hypertexnames=false, % zur korrekten Erstellung der Bookmarks
	    linktocpage % Seitenzahlen anstatt Text im Inhaltsverzeichnis verlinken
	]{hyperref}

% Befehle, die Umlaute ausgeben, f�hren zu Fehlern, wenn sie hyperref als Optionen �bergeben werden
	\hypersetup{
	    pdftitle={\titel \untertitel},
	    pdfauthor={\autor},
	    pdfcreator={\autor},
	    pdfsubject={\titel \untertitel},
	    pdfkeywords={\titel \untertitel},
	}

% fortlaufendes Durchnummerieren der Fu�noten ----------------------------------
	\usepackage{chngcntr}

% f�r lange Tabellen -----------------------------------------------------------
	\usepackage{longtable}
	\usepackage{array}
	\usepackage{ragged2e}
	\usepackage{lscape}

% Spaltendefinition rechtsb�ndig mit definierter Breite ------------------------
	\newcolumntype{w}[1]{>{\raggedleft\hspace{0pt}}p{#1}}

% Formatierung von Listen �ndern -----------------------------------------------
	\usepackage{paralist}

% bei der Definition eigener Befehle ben�tigt
	\usepackage{ifthen}

% definiert u.a. die Befehle \todo und \listoftodos
	\usepackage[]{todonotes}

% sorgt daf�r, dass Leerzeichen hinter parameterlosen Makros nicht als Makroendezeichen interpretiert werden
	\usepackage{xspace}

%Tabellen erstellen
	\usepackage{tabularx}
	\usepackage{booktabs}
	\usepackage{paralist} 

%Einr�cken von Item bei itemize
	\usepackage{enumitem}

%Rotieren von Grafiken
	\usepackage{rotating}

%epigraph
\usepackage{epigraph}
\renewcommand{\epigraphflush}{center}
\renewcommand{\epigraphsize}{\normalsize}
\setlength{\epigraphwidth}{.75\textwidth}
\setlength{\epigraphrule}{0pt}

%Glossar erstllen:
	\usepackage[nonumberlist,nogroupskip,nopostdot,toc]{glossaries}
	\usepackage{glossary-mcols}

%Verschiedene Glossar Varianten:
	%\setglossarystyle{mcolindex}
	%\setglossarystyle{mcolindexhypergroup}
	%\setglossarystyle{mcoltreegroup}
	%\setglossarystyle{mcoltreenoname}
	\setglossarystyle{mcoltreenonamegroup}



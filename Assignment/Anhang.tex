\subsection{Point class listing}
\label{subsec: point-test-listing}
\begin{lstlisting}
package workshop;

import java.awt.geom.AffineTransform;
import java.util.Locale;
import java.util.Objects;

public class Point {
	private double x;
	private double y;
	
	public Point() {
		this.x = 0.0;
		this.y = 0.0;
	}
	
	public Point(double x, double y) {
		assert (!Double.isNaN(x) && !Double.isNaN(y));
		this.x = x;
		this.y = y;
	}
	
	public double getX() {
		return x;
	}
	
	public void setX(double x) {
		assert (!Double.isNaN(x));
		this.x = x;
	}
	
	public double getY() {
		return y;
	}
	
	public void setY(double y) {
		assert (!Double.isNaN(y));
		this.y = y;
	}
	
	@Override
	public boolean equals(Object o) {
		// Easiest case
		if (o == this) {
			return true;
		}
		// if the given object isn't a point, there is no need for further
		// operations
		if (o instanceof Point) {
			Point p = (Point) o;
			// Make use of the hashCode method
			if (p.hashCode() == this.hashCode()) {
				return true;
			}
			return (p.x == this.x && p.y == this.y);
		}
		return false;
	}
	
	@Override
	public int hashCode() {
		// Hash should represent the point's coordinates
		// So two points with the same coordinates should generate the same
		// hash-value
		return Objects.hash(this.x, this.y);
	}
	
	@Override
	public String toString() {
		StringBuilder stringBuilder = new StringBuilder();
		stringBuilder.append("( ");
		stringBuilder.append(String.format(Locale.ROOT, "%+2.4E", this.x));
		stringBuilder.append(", ");
		stringBuilder.append(String.format(Locale.ROOT, "%+2.4E", this.y));
		stringBuilder.append(" )");
		return stringBuilder.toString();
	}
	
	public double norm() {
		// Pythagoras
		return Math.sqrt(this.x * this.x + this.y * this.y);
	}
	
	public void rotate(double theta) throws AngleOutOfRangeException {
		// theta must be an number
		if (Double.isNaN(theta) || theta < -180.0 || theta > 180.0) {
			throw new AngleOutOfRangeException();
		}
		double[] pt = {this.x, this.y};
		AffineTransform.getRotateInstance(Math.toRadians(theta), 0, 0)
		.transform(pt, 0, pt, 0, 1);
		double newX = pt[0];
		double newY = pt[1];
		this.x = newX;
		this.y = newY;
	}
	
	public void displace(Point p) {
		if(p == null){
			return;
		}
		this.x += p.x;
		this.y += p.y;
	}
}
\end{lstlisting}

%%%%%%%%%%%%%%%%%%%%%%%%%%%%%%%%%%%%%%%%%%%%%%%%%%%%%%%%%%%%%%%%%%%%%%%%%%%%%%%%%%%%%%%%%%%%%%%%%%%%%%%%%
%%%%%%%%%%%%%%%%%%%%%%%%%%%%%%%%%%%%%%%%%%%%%%%%%%%%%%%%%%%%%%%%%%%%%%%%%%%%%%%%%%%%%%%%%%%%%%%%%%%%%%%%%
%%%%%%%%%%%%%%%%%%%%%%%%%%%%%%%%%%%%%%%%%%%%%%%%%%%%%%%%%%%%%%%%%%%%%%%%%%%%%%%%%%%%%%%%%%%%%%%%%%%%%%%%%
\pagebreak
\subsection{LineTest class listing}
\label{subsec: line-class-listing}

\begin{lstlisting}
package workshop;

import static org.junit.Assert.*;

import org.junit.Test;

public class LineTest {
	
	//-----------------------------------------------
	// generates some points for different tests
	//-----------------------------------------------
	
	private Point[] getPointsExampleData() {
		Point[] points = new Point[3];
		points[0] = new Point(2.4, -2.4);
		points[1] = new Point(-0.2, 8.9);
		points[2] = new Point(-4.4, -2.4);
		return points;
	}
	
	//-----------------------------------------------
	// Constructor Tests
	//-----------------------------------------------
	
	@Test
	public void testDefaultConstructor() {
		Line line = new Line();
		assertTrue(line.length() == 0);
	}
	
	@Test
	public void testAlternateConstructor() {
		Point[] points = getPointsExampleData();
		Line line = new Line(points);
		assertEquals(3, line.length());
	}
	
	@Test
	public void testAlternateConstructorWithOneNullObject() {
		Point[] points = getPointsExampleData();
		points[0] = null;
		Line line = new Line(points);
		assertEquals(2, line.length());
	}
	
	@Test
	public void testConstructorWithNull() {
		Line line = new Line(null);
		assertTrue(line.length() == 0);
	}
	
	//-----------------------------------------------
	// length() and add() Tests
	//-----------------------------------------------
	
	@Test
	public void testAdd() {
		Line line = new Line();
		line.add(new Point());
		assertEquals(1, line.length());
	}
	
	@Test
	public void testAddWithNull() {
		Line line = new Line();
		line.add(null);
		assertEquals(0, line.length());
	}
	
	public void testLength() {
		Point[] points = getPointsExampleData();
		Line line = new Line(points);
		assertEquals(3, line.length());
	}
	
	public void testLengthAfterAdd() {
		Point[] points = getPointsExampleData();
		Line line = new Line(points);
		assertEquals(3, line.length());
		line.add(new Point(2.3543, 8.345));
		assertEquals(4, line.length());
	}
	
	//-----------------------------------------------
	// equals() Tests
	//-----------------------------------------------
	
	@Test
	public void testEqualsWithExactlySameLineObject() {
		Point[] points = getPointsExampleData();
		Line line = new Line(points);
		assertTrue(line.equals(line));
	}
	
	@Test
	public void testEqualsWithTwoLineObjectsWithSamePoints() {
		Point[] points = getPointsExampleData();
		Line line = new Line(points);
		Line line2 = new Line(points);
		assertTrue(line.equals(line2));
	}
	
	@Test
	public void testEqualsWithTwoLineObjectsWithSamePointsInDiffentOrder() {
		Point[] points = getPointsExampleData();
		Line line = new Line(points);
		Point[] points2 = new Point[3];
		points2[0] = new Point(2.4, -2.4);
		points2[2] = new Point(-0.2, 8.9);
		points2[1] = new Point(-4.4, -2.4);
		Line line2 = new Line(points2);
		assertTrue(line.equals(line2));
	}
	
	@Test
	public void testEqualsWithTwoLinesWithDifferentPoints() {
		Point[] points = getPointsExampleData();
		Line line = new Line(points);
		Point[] points2 = new Point[3];
		points2[0] = new Point(2.3, -1.4);
		points2[2] = new Point(-0.2, 2.9);
		points2[1] = new Point(-42.4, -2.4);
		Line line2 = new Line(points2);
		assertFalse(line.equals(line2));
	}
	
	@Test
	public void testEqualsWithFalseObject() {
		Point[] points = getPointsExampleData();
		Line line = new Line(points);
		// comparing a Line object to a Point object should return false...
		assertFalse(line.equals(new Point(3.9, 4.2)));
	}
	
	@Test
	public void testEqualsWithNull() {
		Point[] points = getPointsExampleData();
		Line line = new Line(points);
		assertFalse(line.equals(null));
	}
	
	//-----------------------------------------------
	// hashCode() Tests
	//-----------------------------------------------
	
	@Test
	public void testHashCodeWithoutPoints() {
		Line line = new Line();
		Line line2 = new Line();
		assertEquals(line.hashCode(), line2.hashCode());
	}
	
	@Test
	public void testHashCodeWithSamePoints() {
		Line line = new Line(getPointsExampleData());
		Line line2 = new Line(getPointsExampleData());
		assertEquals(line.hashCode(), line2.hashCode());
	}
	
	@Test
	public void testHashCodeWithDifferentPoints() {
		Point[] points = getPointsExampleData();
		Line line = new Line(points);
		Line line2 = new Line();
		assertNotEquals(line.hashCode(), line2.hashCode());
	}
	
	//-----------------------------------------------
	// toString() Tests
	//-----------------------------------------------
	
	@Test
	public void testToString() {
		Point[] points = new Point[3];
		points[0] = new Point(0.00000021, 12345.1246);
		points[1] = new Point(-0.2556785, 1.1246365);
		points[2] = new Point(-235.25853467, -1.1246);
		Line line = new Line(points);
		String expectedOutout = "(( +2.1000E-07, +1.2345E+04 ),\n ( -2.5568E-01, +1.1246E+00 ),\n ( -2.3526E+02, -1.1246E+00 ))";
		assertEquals(expectedOutout, line.toString());
	}
	
	//-----------------------------------------------
	// isValid() Tests
	//-----------------------------------------------
	
	@Test
	public void testIsValidWithZeroPoints() {
		Line line = new Line();
		assertFalse(line.isValid());
	}
	
	@Test
	public void testIsValidWithOnePoint() {
		Line line = new Line();
		line.add(new Point(2.1, 4.3));
		assertFalse(line.isValid());
	}
	
	@Test
	public void testIsValidWithManyPoints() throws RegressionFailedException {
		Point[] points = getPointsExampleData();
		Line line = new Line(points);
		assertTrue(line.isValid());
	}
	
	@Test
	public void testIsValidWithTwoPointsWithSameCoordinates() {
		Point[] points = new Point[2];
		points[0] = new Point(2.1, 2.1);
		points[1] = new Point(2.1, 2.1);
		Line line = new Line(points);
		assertFalse(line.isValid());
	}
	
	@Test
	public void testIsValidWithTwoPointsWithSameXCoordinates() {
		Point[] points = new Point[2];
		points[0] = new Point(2.1, 23.34);
		points[1] = new Point(2.1, 21.23);
		Line line = new Line(points);
		assertFalse(line.isValid());
	}
	//-----------------------------------------------
	// Slope Tests
	//-----------------------------------------------
	
	@Test
	public void testSlopeWithTwoValidPoints() throws RegressionFailedException{
		// very simple tests
		Point[] points = new Point[2];
		points[0] = new Point(0, 0);
		points[1] = new Point(1, 1);
		Line line = new Line(points);
		double slope = line.slope();
		assertEquals(1, slope, 0);
	}
	
	@Test
	public void testSlopeWithValidPoints() throws RegressionFailedException{
		// more complex parameters
		Point[] points = new Point[6];
		points[0] = new Point(0.2, 0);
		points[1] = new Point(1, 3.12);
		points[2] = new Point(3.23, 1);
		points[3] = new Point(-1.5, 1.3);
		points[4] = new Point(4.9, 12.2);
		points[5] = new Point(3.32, 2);
		Line line = new Line(points);
		double slope = line.slope();
		assertEquals(1.2, slope, 0.03);
	}
	
	@Test(expected = RegressionFailedException.class)
		public void testSlopeWithSamePoints() throws RegressionFailedException{
		Point[] points = new Point[2];
		points[0] = new Point(0, 0);
		points[1] = new Point(0, 0);
		// these points have the same coordinates. 
		// So the slope can't be calculated with these two.
		Line line = new Line(points);
		line.slope();
	}
	
	//-----------------------------------------------
	// Intercept Tests
	//-----------------------------------------------
	
	@Test
	public void testInterceptIWithValidPoints() throws RegressionFailedException{
		Point[] points = new Point[2];
		points[0] = new Point(0, 0);
		points[1] = new Point(1, 1);
		Line line = new Line(points);
		double intercept = line.intercept();
		assertEquals(0, intercept, 0);
	}
	
	@Test(expected = RegressionFailedException.class)
	public void testInterceptIWithInvalidPoints() throws RegressionFailedException{
		Point[] points = new Point[2];
		points[0] = new Point(0, 0);
		points[1] = new Point(0, 1);
		// with these points the line lies on the y-axis. therefore slope can't be calculated. 
		// And without slope, there is no intercept.
		Line line = new Line(points);
		line.intercept();
	}
}
\end{lstlisting}

\pagebreak
%%%%%%%%%%%%%%%%%%%%%%%%%%%%%%%%%%%%%%%%%%%%%%%%%%%%%%%%%%%%%%%%%%%%%%%%%%%%%%%%%%%%%%%%%%%%%%%%%%%%%%%%%
%%%%%%%%%%%%%%%%%%%%%%%%%%%%%%%%%%%%%%%%%%%%%%%%%%%%%%%%%%%%%%%%%%%%%%%%%%%%%%%%%%%%%%%%%%%%%%%%%%%%%%%%%
%%%%%%%%%%%%%%%%%%%%%%%%%%%%%%%%%%%%%%%%%%%%%%%%%%%%%%%%%%%%%%%%%%%%%%%%%%%%%%%%%%%%%%%%%%%%%%%%%%%%%%%%%
\subsection{LinearRegressionProgram class listing}
\label{subsec: analyser-class-listing}

\begin{lstlisting}
package workshop;

import java.io.IOException;
import java.util.ArrayList;
import java.util.Collection;
import java.util.Date;
import java.util.HashMap;
import java.util.List;
import java.util.Map;

public class LinearRegressionProgram {
	
	// Files 
	private static final String FILENAME_DATA_SHORT = "C:\\Users\\User\\Desktop\\TDD\\TDDAssignment\\TDD\\src\\main\\java\\workshop\\data_short.dat";
	private static final String FILENAME_DATA_LONG = "C:\\Users\\User\\Desktop\\TDD\\TDDAssignment\\TDD\\src\\main\\java\\workshop\\data_long.dat";
	private static final String FILENAME_READER_EXCEL = "ReaderData.xlsx"; 
	private static final String FILENAME_ANALYZER_EXCEL = "AnalyzerData.xlsx"; 
	
	// Data to calculate
	private static int inValidLineCount = 0;
	private static int validLineCount = 0;
	private static double averagePointsPerValidLine = 0;
	private static double averageSlopeForValidLines = 0;
	private static double averageInterceptForValidLines = 0;
	private static List<Double> slopes = new ArrayList<Double>();
	private static List<Double> interceptionsForY = new ArrayList<Double>();
	private static double interceptsStandardDeviation = 0;
	private static double slopesStandardDeviation = 0;
	private static Map<Integer, List<Long>> analyzerDataForCalculations;
	
	// Times stopped in milliseconds
	private static long overallTime;
	private static long readTime;
	private static long analyzeTime;
	private static long printAndParseResultsTime;
	
	public static void main(String[] args) {
		Date startOverallTimestamp = new Date();
		analyseDatFile(FILENAME_DATA_LONG);
		Date endOverallTimestamp = new Date();
		overallTime = endOverallTimestamp.getTime() - startOverallTimestamp.getTime();
		printTimes(FILENAME_DATA_LONG);
	}
	
	private static void analyseDatFile(final String fileName) {
		analyzerDataForCalculations = new HashMap<Integer, List<Long>>();
		Date startReadTimestamp = new Date();
		Collection<Line> linesFromFile = LineReader.readAllLinesFromFile(fileName);
		Date endReadTimestamp = new Date();
		readTime = endReadTimestamp.getTime() - startReadTimestamp.getTime();
		Date startAnalyzeTimestamp = new Date();
		calculateStatistics(linesFromFile);
		Date endAnalyzeTimestamp = new Date();
		analyzeTime = endAnalyzeTimestamp.getTime() - startAnalyzeTimestamp.getTime();
		Date startPrintAndParseResultsTimestamp = new Date();
		printStatistics(fileName);
		Map<Integer, Long> normalizedReaderData = normalizeTimeData(LineReader.getAnalyzerData());
		Map<Integer, Long> normalizedAnalyzerData = normalizeTimeData(analyzerDataForCalculations);
		try {
			ExcelWriter.writeExcelOutput(normalizedReaderData, FILENAME_READER_EXCEL);
			ExcelWriter.writeExcelOutput(normalizedAnalyzerData, FILENAME_ANALYZER_EXCEL);
		} catch (IOException e) {
			System.err.println("Excel Creation failed !");
			e.printStackTrace();
		}
		Date endPrintAndParseResultsTimestamp = new Date();
		printAndParseResultsTime = endPrintAndParseResultsTimestamp.getTime() - startPrintAndParseResultsTimestamp.getTime();		
	}
	
	// print-method for statistics outputs
	private static void printStatistics(String fileName) {
		System.out.println("\nStatistics for file: "+fileName);
		System.out.println("Total number of valid lines: " + validLineCount);
		System.out.println("Total number of invalid lines: " + inValidLineCount);
		System.out.println("Average number of points per valid line: " + averagePointsPerValidLine);
		System.out.println("Average Slope for Valid Lines: " + averageSlopeForValidLines);
		System.out.println("with standard deviation: " + slopesStandardDeviation);
		System.out.println("Average intercept for valid lines: " + averageInterceptForValidLines);
		System.out.println("with standard deviation: " + interceptsStandardDeviation);
	}
	
	// print-method for time outputs
	private static void printTimes(String fileName) {
		System.out.println("\n\nTimes stopped for file: "+fileName);
		System.out.println("programm runtime in milliseconds: " + overallTime);
		System.out.println("reading dat-files runtime in milliseconds: " + readTime);
		System.out.println("analyzing data runtime in milliseconds: " + analyzeTime);
		System.out.println("printing and parsing results runtime in milliseconds: " + printAndParseResultsTime);
	}
	
	// analyses all lines for required data
	private static void calculateStatistics(Collection<Line> linesFromFile) {
		double slopeSum = 0.0;
		double interceptSum = 0.0;
		int pointSum = 0;
		
		for (Line line : linesFromFile) {
			Date startLine = new Date();
			if (line.isValid()) {
				validLineCount++;
				pointSum += line.length();
				try {
					double slope = line.slope();
					slopeSum += slope;
					slopes.add(slope);
					double intercept = line.intercept();
					interceptSum += intercept;
					interceptionsForY.add(intercept);
				} catch (RegressionFailedException e) {
					e.printStackTrace();
				}
			} else {
				inValidLineCount++;
			}
			Date endLine = new Date();
			// Get the amount of time it took, to calculate statistics for a line.
			long timeToReadLine = endLine.getTime() - startLine.getTime();
			if (analyzerDataForCalculations.containsKey(line.length())) {
				analyzerDataForCalculations.get(line.length()).add(timeToReadLine);
			} else {
				List<Long> timeList = new ArrayList<>();
				timeList.add(timeToReadLine);
				analyzerDataForCalculations.put(line.length(), timeList);
			}
		}
		// Dividing by 0 destroys reality
		if (validLineCount > 0) {
			averagePointsPerValidLine = (double) pointSum / validLineCount;
			averageSlopeForValidLines = (double) slopeSum / validLineCount;
			averageInterceptForValidLines = (double) interceptSum / validLineCount;
		}
		slopesStandardDeviation = calculateDeviation(slopes);
		interceptsStandardDeviation = calculateDeviation(interceptionsForY);
	}
	
	// Calculates the mean for a List of double-values
	private static double calculateMean(List<Double> list) {
		return list.stream().mapToDouble(p -> p.doubleValue()).sum() / list.size();
	}
	
	// Calculates the standard deviation for a list of double-values
	private static double calculateDeviation(List<Double> list) {
		double temp = 0;
		for (Double double1 : list) {
			temp += Math.pow((double1.doubleValue() - calculateMean(list)), 2);
		}
		return Math.sqrt(temp / list.size());
	}
	
	// Gets the avarage for the time it took to process a line with a certain 
	// amount of points from raw timestamp-data
	private static Map<Integer, Long> normalizeTimeData(Map<Integer, List<Long>> data){
		Map<Integer, Long> normalizedData = new HashMap<Integer, Long>();
		for (Integer key : data.keySet()){
			List<Long> stoppedTimes = data.get(key);
			long avarage = 0;
			long sum = 0;
			for (Long time : stoppedTimes){
				sum += time;
			}
			if(stoppedTimes.size() != 0){
				avarage = (long) sum / stoppedTimes.size();
			}
			normalizedData.put(key, avarage);
		}
		return normalizedData;
	}
}
\end{lstlisting}

%%%%%%%%%%%%%%%%%%%%%%%%%%%%%%%%%%%%%%%%%%%%%%%%%%%%%%%%%%%%%%%%%%%%%%%%%%%%%%%%%%%%%%%%%%%%%%%%%%%%%%%%%
%%%%%%%%%%%%%%%%%%%%%%%%%%%%%%%%%%%%%%%%%%%%%%%%%%%%%%%%%%%%%%%%%%%%%%%%%%%%%%%%%%%%%%%%%%%%%%%%%%%%%%%%%
%%%%%%%%%%%%%%%%%%%%%%%%%%%%%%%%%%%%%%%%%%%%%%%%%%%%%%%%%%%%%%%%%%%%%%%%%%%%%%%%%%%%%%%%%%%%%%%%%%%%%%%%%
\pagebreak
\subsection{Line Reader listing}
\label{subsec: LineReader-listing}
\begin{lstlisting}
package workshop;

import java.util.ArrayList;
import java.util.Collection;
import java.util.Date;
import java.util.HashMap;
import java.util.List;
import java.util.Map;

import uk.ac.brunel.ee.lineRead;
import uk.ac.brunel.ee.UnreadException;
import uk.ac.brunel.ee.RereadException;

public class LineReader {
	
	private static Map<Integer, List<Long>> timeData = new HashMap<Integer, List<Long>>();
	
	
	public static Map<Integer, List<Long>> getAnalyzerData() {
	Map<Integer, List<Long>> analyzerDataCopy = timeData;
	return analyzerDataCopy;
	}
	
	/**
	* @param fileName
	*            the filename
	*/
	public static Collection<Line> readAllLinesFromFile(String fileName) {
		// new Analysis
		timeData = new HashMap<Integer, List<Long>>();
		
		List<Line> lines = new ArrayList<Line>();
		
		Date start = new Date();
		
		// Open the file and initialise
		lineRead reader = new lineRead(fileName);
		
		// Loop over all the lines in the data set
		while (reader.nextLine()) {
			Date startLineTimeStamp = new Date();
			Line line = new Line();
			boolean np = true;
			// Loop over all the points associated with the current line
			while (np) {
				try {
					np = reader.nextPoint();
				} catch (UnreadException UE) {
					System.out.println(UE);
					System.exit(0);
				}
				// If there is another point read it.
				if (np) {
					try {
						double x = reader.getX();
						double y = reader.getY();
						Point point = new Point(x, y);
						line.add(point);
					} catch (RereadException RE) {
						System.out.println(RE);
						System.exit(0);
					}
				}
			}
			lines.add(line);
			Date endLineTimeStamp = new Date();
			// Get the duration it took, to read a line in milliseconds.
			long timeToReadLine = endLineTimeStamp.getTime() - startLineTimeStamp.getTime();
			if (timeData.containsKey(line.length())) {
				timeData.get(line.length()).add(timeToReadLine);
			} else {
				List<Long> timeList = new ArrayList<>();
				timeList.add(timeToReadLine);
				timeData.put(line.length(), timeList);
			}
		}
		// Sort out the summary of the run
		Date end = new Date();
		long begin = start.getTime();
		long fin = end.getTime();
		System.out.println("run time is " + (fin - begin) + " milliseconds");
		return lines;
	}
}
\end{lstlisting}

%%%%%%%%%%%%%%%%%%%%%%%%%%%%%%%%%%%%%%%%%%%%%%%%%%%%%%%%%%%%%%%%%%%%%%%%%%%%%%%%%%%%%%%%%%%%%%%%%%%%%%%%%
%%%%%%%%%%%%%%%%%%%%%%%%%%%%%%%%%%%%%%%%%%%%%%%%%%%%%%%%%%%%%%%%%%%%%%%%%%%%%%%%%%%%%%%%%%%%%%%%%%%%%%%%%
%%%%%%%%%%%%%%%%%%%%%%%%%%%%%%%%%%%%%%%%%%%%%%%%%%%%%%%%%%%%%%%%%%%%%%%%%%%%%%%%%%%%%%%%%%%%%%%%%%%%%%%%%
\pagebreak
\subsection{Excel Writer listing}
\label{subsec: ExcelWriter-listing}
\begin{lstlisting}
package workshop;

import java.io.File;
import java.io.FileInputStream;
import java.io.FileOutputStream;
import java.io.IOException;
import java.io.InputStream;
import java.util.ArrayList;
import java.util.HashMap;
import java.util.List;
import java.util.Map;
import java.util.Set;

import org.apache.poi.ss.usermodel.Cell;
import org.apache.poi.xssf.usermodel.XSSFRow;
import org.apache.poi.xssf.usermodel.XSSFSheet;
import org.apache.poi.xssf.usermodel.XSSFWorkbook;

public class ExcelWriter {

	static XSSFWorkbook workbook;

	public void getList(List<Integer> numbersOfPoints, List<Long> stoppedTimes) {
		Map<Integer, List<?>> hashmap = new HashMap<Integer, List<?>>();
		hashmap.put(0, numbersOfPoints);
		hashmap.put(1, stoppedTimes);
		System.out.println("hashmap : " + hashmap);
		Set<Integer> keyset = hashmap.keySet();
		int rownum = 0;
		int cellnum = 0;
		XSSFSheet sheet = workbook.getSheetAt(0);
		rownum = 0;
		for (Integer key : keyset) {
			List<?> nameList = hashmap.get(key);
			for (Object s : nameList) {
				XSSFRow row = sheet.getRow(rownum++);
				Cell cell = row.getCell(cellnum);
				if (null != cell) {
					if (s instanceof Long) {
						cell.setCellValue((Long) s);
					}
					if (s instanceof Integer) {
						cell.setCellValue((Integer) s);
					}
				}
			}
			cellnum++;
			rownum = 0;
		}
	}
	
	public static void writeExcelOutput(Map<Integer, Long> normalizedData, String FileName) throws IOException {
		workbook = new XSSFWorkbook();
		List<Integer> keys = new ArrayList<>();
		List<Long> values = new ArrayList<>();
		for (Integer key : normalizedData.keySet()) {
			keys.add(key);
			values.add(normalizedData.get(key));
		}
		ExcelWriter writer = new ExcelWriter();
		FileOutputStream out = new FileOutputStream(new File(FileName));
		XSSFSheet sheet = workbook.createSheet();
		for (int i = 0; i < normalizedData.keySet().size(); i++) {
			XSSFRow row = sheet.createRow(i);
			for (int j = 0; j < 2; j++)
				row.createCell(j);
		}
		workbook.write(out);
		out.close();
		InputStream inp = new FileInputStream(new File(FileName));
		workbook = new XSSFWorkbook(inp);
		writer.getList(keys, values);
		out = new FileOutputStream(new File(FileName));
		workbook.write(out);
		out.close();
		inp.close();
	}
}
\end{lstlisting}
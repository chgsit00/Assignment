\lstset{language=Java, numbers=left, numberstyle=\tiny, stepnumber=2, numbersep=5pt}
\chapter{Introduction}
\label{intro}
The following reports refers to the Computer network assignment and is structured into three parts. The first part's topic is an analysis of the network protocols ICMP and IP (both v4), while the seconds part covers the exercises related to TCP. The final chapter describes the exercises for the new versions of ICMP and IP (v6). These exercises were done together with my lab-partner Antonio Parotta.

%%%%%%%%%%%%%%%%%%%%%%%%%%%%%%%%%%%%%%%%%%%%%%%%%%%%%%%%%%%%%%%%%%%%%%%%%%%%%%%%%%%%%%%%%%%%%%%%%%%%%%%%%
%%%%%%%%%%%%%%%%%%%%%%%%%%%%%%%%%%%%%%%%%%%%%%%%%%%%%%%%%%%%%%%%%%%%%%%%%%%%%%%%%%%%%%%%%%%%%%%%%%%%%%%%%
%%%%%%%%%%%%%%%%%%%%%%%%%%%%%%%%%%%%%%%%%%%%%%%%%%%%%%%%%%%%%%%%%%%%%%%%%%%%%%%%%%%%%%%%%%%%%%%%%%%%%%%%%
\chapter{IP/ICMP analysis}
\label{ipv4}

In this first part of the laboratory the program Wireshark was used to capture and analyse packages of different network protocols. The traffic was generated by PING-commands to send the observable packages from one lab PC to another. The following network protocols were analyzed:
\begin{itemize}
	\item Internet Protocol version 4 (IPv4)
	\item Internet Control Message Protocol version 4 (ICMPv4)
	\item Address Resolution Protocol (ARP)
	\item Carrier Sense Multiple Access/Collision Detection (CSMA/CD)
\end{itemize}
To understand how these protocols work and to be able to explain how they behave in different situations, having a look on the protocol's headers is necessary. \\\\
The PING-commands generate packages consisting of different protocol headers and transferable data. Each Ping is transformed into an Ethernet frame containing the IP and ICMP headers. Table \ref{abstract-icmp-header} is an representation of the basic ICMP Header while Table \ref{icmp-echo-request-header} shows the header for the echo request/reply packages that can be observed via Wireshark when executing the PING-commands.

\begin{table}[]
	\centering
	\label{abstract-icmp-header}
	\begin{tabular}{|c|c|c|c|c|}
		\hline
		\textbf{bits}  & 0-7  & 8-15 & 16-23         & 24-31         \\ \hline
		\textbf{bytes} & 1    & 2    & 3             & 4             \\ \hline
		offset 0       & Type & Code & \multicolumn{2}{c|}{Checksum} \\ \hline
		offset 32      & \multicolumn{4}{c|}{Data}                   \\ \hline
	\end{tabular}
	\caption{ICMP header}
\end{table}

\begin{table}[]
	\centering
	\label{icmp-echo-request-header}
	\begin{tabular}{|c|c|c|c|c|}
		\hline
		\textbf{bits}                   & 0-7            & 8-15           & 16-23             & 24-31            \\ \hline
		\textbf{bytes}                  & 1              & 2              & 3                 & 4                \\ \hline
		offset 0                        & Type           & Code           & \multicolumn{2}{c|}{Checksum}        \\ \hline
		\multicolumn{1}{|l|}{offset 32} & \multicolumn{2}{c|}{Identifier} & \multicolumn{2}{c|}{Sequence Number} \\ \hline
		offset 64                       & \multicolumn{4}{c|}{data}                                              \\ \hline
	\end{tabular}
	\caption{ICMP type 8 echo request/reply packet }
	
\end{table}

Table \ref{ipv4-header} shows the header for the Internet Protocol v4. Noteable here are the entered destination address as well as the source address of the sender. The Time to Live is also an important segment of the header, which will be significant later on for an specific PING-Command. IP provides the possibility to specify options for the transfered packet. This will also be used in one of the PING-Commands.

\begin{table}[]
	\centering
	\label{ipv4-header}
	\begin{tabular}{|c|c|c|c|c|c|c|c|c|}
		\hline
		bits       & 0-3               & 4-7           & 8-11             & 12-15             & 16-18  & 19-23      & 24-27      & 28-31      \\ \hline
		bytes      & 1                 & 2             & 3                & 4                 & 5      & 6          & 7          & 8          \\ \hline
		offset 0   & Version           & IHL           & \multicolumn{2}{c|}{Type of Service} & \multicolumn{4}{c|}{Total Length}             \\ \hline
		offset 32  & \multicolumn{4}{c|}{Identification}                                      & Flags  & \multicolumn{3}{c|}{Fragment Offset} \\ \hline
		offset 64  & \multicolumn{2}{c|}{Time to Live} & \multicolumn{2}{c|}{Protocol}        & \multicolumn{4}{c|}{Header Checksum}          \\ \hline
		offset 96  & \multicolumn{8}{c|}{Source Address}                                                                                      \\ \hline
		offset 128 & \multicolumn{8}{c|}{Destination Address}                                                                                 \\ \hline
		offset 160 & \multicolumn{8}{c|}{Options}                                                                                             \\ \hline
	\end{tabular}
	\caption{IPv4 Header}
\end{table}

Table \ref{ethernet} shows the abstract Ethernet II frame. This frame contains the MAC-addresses for source and destination, a type segment as well as the checksum for the frame. Interesting here is the Payload field. This segments contains the headers for ICMP and IP as well as the transferable data. The maximal size for this segment is 1500 bytes for one packet. But because is must contain the headers for each networking protocol (ICMP and IP), it can't be fully occupied by transferable data. This is why the Maximum Transmission Unit (MTU) is smaller. It is only 1472 bytes, because the size of the headers must be subtracted from the payload field.
$ \text{MTU } = \text{ Payload } - \text{ IP Header } - \text{ ICMP Header }$\\
$ 1500\text{ byte } - 20 \text{ byte } - 8 \text{ byte } = 1472 \text{ byte }$

\begin{table}[]
	\centering
	\label{ethernet}
	\begin{tabular}{|l|l|l|l|l|l|}
		\hline
		Size in bit    & 24                  & 24             & 8    & 184-6000       & 16             \\ \hline
		Size in byte   & 6                   & 6              & 2    & 46 - 1500      & 4              \\ \hline
		Frame segments & Destination Address & Source Address & Type & Payload (Data) & FCS \\ \hline
	\end{tabular}
	\caption{Ethernet II frame}
\end{table}

\section{Node configuration}

\section{Subnet internal IP Destination}
\subsection{a) Basic PING command}
\subsection{b) PING command with large data package}
\subsection{c) PING command with 'don't fragment' flag}

\section{Subnet external IP Destination}
\subsection{d) Basic PING command with destination in another subnet}
\subsection{e) PING command with reduced 'time to live'}
\subsection{f) PING command with timestamps}

\section{ARP analysis}
\subsection{a) Deleting the ARP cache}
\subsection{b) Shutting down one PC}
\subsection{c) Reconnect after Reboot}

\section{IP multicast addressing}
%%%%%%%%%%%%%%%%%%%%%%%%%%%%%%%%%%%%%%%%%%%%%%%%%%%%%%%%%%%%%%%%%%%%%%%%%%%%%%%%%%%%%%%%%%%%%%%%%%%%%%%%%
%%%%%%%%%%%%%%%%%%%%%%%%%%%%%%%%%%%%%%%%%%%%%%%%%%%%%%%%%%%%%%%%%%%%%%%%%%%%%%%%%%%%%%%%%%%%%%%%%%%%%%%%%
%%%%%%%%%%%%%%%%%%%%%%%%%%%%%%%%%%%%%%%%%%%%%%%%%%%%%%%%%%%%%%%%%%%%%%%%%%%%%%%%%%%%%%%%%%%%%%%%%%%%%%%%%
\chapter{TCP analysis}
\label{tcp}
\section{Traffic generator handling}

\section{Simple TCP Communication}
\subsection{Connection establishment}
\subsection{Data transfer}
\subsection{Connection release}

\section{TCP flow control}

\section{TCP transmission error recovery/abort}

\section{TCP protocol errors (synchronization errors)}

%%%%%%%%%%%%%%%%%%%%%%%%%%%%%%%%%%%%%%%%%%%%%%%%%%%%%%%%%%%%%%%%%%%%%%%%%%%%%%%%%%%%%%%%%%%%%%%%%%%%%%%%%
%%%%%%%%%%%%%%%%%%%%%%%%%%%%%%%%%%%%%%%%%%%%%%%%%%%%%%%%%%%%%%%%%%%%%%%%%%%%%%%%%%%%%%%%%%%%%%%%%%%%%%%%%
%%%%%%%%%%%%%%%%%%%%%%%%%%%%%%%%%%%%%%%%%%%%%%%%%%%%%%%%%%%%%%%%%%%%%%%%%%%%%%%%%%%%%%%%%%%%%%%%%%%%%%%%%
\chapter{IPv6/ICMPv6 analysis}
\label{ipv6}
\section{Node configuration}
\subsection{IPv4 and IPv6 configuration}
\subsection{interfaces for IPv6}

\section{PING commands}
\subsection{a) Basic ICMPv6 PING command}
\subsection{b) ICMPv6 PING command with large data package}
\subsection{c) Rebooting PC}
\subsection{d) Enforcing Neighbor discovery}
\subsection{e) ICMPv6 PING command with destination in another subnet}
\subsection{f) PING to a remote tunnel end}

%%%%%%%%%%%%%%%%%%%%%%%%%%%%%%%%%%%%%%%%%%%%%%%%%%%%%%%%%%%%%%%%%%%%%%%%%%%%%%%%%%%%%%%%%%%%%%%%%%%%%%%%%
%%%%%%%%%%%%%%%%%%%%%%%%%%%%%%%%%%%%%%%%%%%%%%%%%%%%%%%%%%%%%%%%%%%%%%%%%%%%%%%%%%%%%%%%%%%%%%%%%%%%%%%%%
%%%%%%%%%%%%%%%%%%%%%%%%%%%%%%%%%%%%%%%%%%%%%%%%%%%%%%%%%%%%%%%%%%%%%%%%%%%%%%%%%%%%%%%%%%%%%%%%%%%%%%%%%
\chapter{Conclusion}
\label{conclusion}
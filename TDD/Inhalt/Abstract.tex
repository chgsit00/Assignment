\section*{Abstract}
\label{sec:Zusammenfassung}

Der vorliegende Praxisbericht, der mehrere Projekte im Themengebiet ``Lokalisierung`` und ``Artificial Intelligence for Robotics`` behandelt, wurde im Zeitraum des 01.09.2013 - 28.02.2014 in der Firma IT-Designers GmbH erstellt.

Ziel war die Einarbeitung in das Thema, sowie dem dazu verwendetem Framework ROS (Robotic Operating System). Dazu wurde auch ein Programm zur Erkennung der Fahrtrichtung eines Testfahrzeug, mit Hilfe von ROS, erstellt. Das Framework und seine Arbeitweise wird beschrieben und anhand grafischer Darstellung nochmals in einer Praxisanwendung gezeigt.

Bei dem darauf folgendem Projekt wurden die Unterschiede zweier Algorithmen, zur Lokalisierung anhand von Messdaten, ausgearbeitet. Dazu werden beide Algorithmen aufgelistet und erl�utert. 
Des Weiteren wird eine bildliche Darstellung des Ablaufes der Algorithmen aufgef�hrt. Eine bildliche Darstellung des Ablaufs vereinfacht das Verst�ndnis dieser Algorithmen. Nach der ausf�hrlichen Auff�hrung der Algorithmen wird ein Fazit gezogen, sowie die Vor- und Nachteile beider aufgelistet.



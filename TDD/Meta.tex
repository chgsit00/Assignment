% Meta-Informationen -----------------------------------------------------------
%   Definition von globalen Parametern, die im gesamten Dokument verwendet
%   werden k�nnen (z.B auf dem Deckblatt etc.).
%
%   ACHTUNG: Wenn die Texte Umlaute oder ein Esszet enthalten, muss der folgende
%            Befehl bereits an dieser Stelle aktiviert werden:
%            \usepackage[latin1]{inputenc}
% ------------------------------------------------------------------------------
\newcommand{\titel}{Assignment}
\newcommand{\untertitel}{Test Driven Development}
\newcommand{\art}{Informatik}
\newcommand{\fachgebiet}{Informatik}
\newcommand{\autor}{Mateusz Szlek}
\newcommand{\studienbereich}{Informatik}
\newcommand{\matrikelnr}{327977}
\newcommand{\erstgutachter}{Prof. Dr. Pado}
\newcommand{\zweitgutachter}{Tobias Langjahr M.Sc.}
\newcommand{\jahr}{2016}
\newcommand{\ort}{Esslingen am Neckar}
\newcommand{\logo}{Brunel_University_Logo.png}


%K�rzel f�r h�ufige Fachbegriffe 
\newcommand{\SLAM}{Simultaneous Localization and Mapping }
\newcommand{\ROS}{Robot Operating System }
\newcommand{\AI}{Artificial Intelligence for Robotics }


%\newcommand{\RN}{\gls{real number}} 
\newcommand{\pose}{\gls{Pose} }

%Nummerierung der parts von Roman auf Alpha setzten
\renewcommand\thepart{\Alph{part}}
\renewcommand\thepart{\Alph{part}}